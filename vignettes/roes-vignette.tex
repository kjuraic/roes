\documentclass[]{article}
\usepackage{lmodern}
\usepackage{amssymb,amsmath}
\usepackage{ifxetex,ifluatex}
\usepackage{fixltx2e} % provides \textsubscript
\ifnum 0\ifxetex 1\fi\ifluatex 1\fi=0 % if pdftex
  \usepackage[T1]{fontenc}
  \usepackage[utf8]{inputenc}
\else % if luatex or xelatex
  \ifxetex
    \usepackage{mathspec}
  \else
    \usepackage{fontspec}
  \fi
  \defaultfontfeatures{Ligatures=TeX,Scale=MatchLowercase}
\fi
% use upquote if available, for straight quotes in verbatim environments
\IfFileExists{upquote.sty}{\usepackage{upquote}}{}
% use microtype if available
\IfFileExists{microtype.sty}{%
\usepackage{microtype}
\UseMicrotypeSet[protrusion]{basicmath} % disable protrusion for tt fonts
}{}
\usepackage[margin=1in]{geometry}
\usepackage{hyperref}
\hypersetup{unicode=true,
            pdftitle={roes - Optical Emission Spectroscopy Data Analysis},
            pdfauthor={Krunoslav Juraic},
            pdfborder={0 0 0},
            breaklinks=true}
\urlstyle{same}  % don't use monospace font for urls
\usepackage{longtable,booktabs}
\usepackage{graphicx,grffile}
\makeatletter
\def\maxwidth{\ifdim\Gin@nat@width>\linewidth\linewidth\else\Gin@nat@width\fi}
\def\maxheight{\ifdim\Gin@nat@height>\textheight\textheight\else\Gin@nat@height\fi}
\makeatother
% Scale images if necessary, so that they will not overflow the page
% margins by default, and it is still possible to overwrite the defaults
% using explicit options in \includegraphics[width, height, ...]{}
\setkeys{Gin}{width=\maxwidth,height=\maxheight,keepaspectratio}
\IfFileExists{parskip.sty}{%
\usepackage{parskip}
}{% else
\setlength{\parindent}{0pt}
\setlength{\parskip}{6pt plus 2pt minus 1pt}
}
\setlength{\emergencystretch}{3em}  % prevent overfull lines
\providecommand{\tightlist}{%
  \setlength{\itemsep}{0pt}\setlength{\parskip}{0pt}}
\setcounter{secnumdepth}{5}
% Redefines (sub)paragraphs to behave more like sections
\ifx\paragraph\undefined\else
\let\oldparagraph\paragraph
\renewcommand{\paragraph}[1]{\oldparagraph{#1}\mbox{}}
\fi
\ifx\subparagraph\undefined\else
\let\oldsubparagraph\subparagraph
\renewcommand{\subparagraph}[1]{\oldsubparagraph{#1}\mbox{}}
\fi

%%% Use protect on footnotes to avoid problems with footnotes in titles
\let\rmarkdownfootnote\footnote%
\def\footnote{\protect\rmarkdownfootnote}

%%% Change title format to be more compact
\usepackage{titling}

% Create subtitle command for use in maketitle
\newcommand{\subtitle}[1]{
  \posttitle{
    \begin{center}\large#1\end{center}
    }
}

\setlength{\droptitle}{-2em}
  \title{roes - Optical Emission Spectroscopy Data Analysis}
  \pretitle{\vspace{\droptitle}\centering\huge}
  \posttitle{\par}
  \author{Krunoslav Juraic}
  \preauthor{\centering\large\emph}
  \postauthor{\par}
  \predate{\centering\large\emph}
  \postdate{\par}
  \date{2018-04-30}


\begin{document}
\maketitle

Roes vignette gives one example of Optical Emission Spectroscopy data
analysis. Analysis can be devided in 3 steps:

\begin{itemize}
\tightlist
\item
  OES data import
\item
  OES data cleaning
\item
  OES data comparison with NIST database (spectral lines recognition)
\end{itemize}

Package can be installed directly from github:

Before use package should be loaded:

\begin{verbatim}
library(roes)
\end{verbatim}

\subsection{Data import}\label{data-import}

In this example we are using Ocean Optics HR4000 spectrometer and data
saved in ASCII files by Oean Optics Spectra Suite spectrometer. There
are two Ocean Optics ASCII data files: with header or without header.

\subsection{Data cleaning}\label{data-cleaning}

\subsection{NIST spectral lines
database}\label{nist-spectral-lines-database}

\subsection{Figures}\label{figures}

The figure sizes have been customised so that you can easily put two
images side-by-side.

\begin{verbatim}
plot(1:10)
plot(10:1)
\end{verbatim}

\includegraphics{roes-vignette_files/figure-latex/unnamed-chunk-3-1.pdf}
\includegraphics{roes-vignette_files/figure-latex/unnamed-chunk-3-2.pdf}

You can enable figure captions by \texttt{fig\_caption:\ yes} in YAML:

\begin{verbatim}
output:
  rmarkdown::html_vignette:
    fig_caption: yes
\end{verbatim}

Then you can use the chunk option
\texttt{fig.cap\ =\ "Your\ figure\ caption."} in \textbf{knitr}.

\subsection{More Examples}\label{more-examples}

You can write math expressions, e.g. \(Y = X\beta + \epsilon\),
footnotes\footnote{A footnote here.}, and tables, e.g.~using
\texttt{knitr::kable()}.

\begin{longtable}[]{@{}lrrrrrrrrrrr@{}}
\toprule
& mpg & cyl & disp & hp & drat & wt & qsec & vs & am & gear &
carb\tabularnewline
\midrule
\endhead
Mazda RX4 & 21.0 & 6 & 160.0 & 110 & 3.90 & 2.620 & 16.46 & 0 & 1 & 4 &
4\tabularnewline
Mazda RX4 Wag & 21.0 & 6 & 160.0 & 110 & 3.90 & 2.875 & 17.02 & 0 & 1 &
4 & 4\tabularnewline
Datsun 710 & 22.8 & 4 & 108.0 & 93 & 3.85 & 2.320 & 18.61 & 1 & 1 & 4 &
1\tabularnewline
Hornet 4 Drive & 21.4 & 6 & 258.0 & 110 & 3.08 & 3.215 & 19.44 & 1 & 0 &
3 & 1\tabularnewline
Hornet Sportabout & 18.7 & 8 & 360.0 & 175 & 3.15 & 3.440 & 17.02 & 0 &
0 & 3 & 2\tabularnewline
Valiant & 18.1 & 6 & 225.0 & 105 & 2.76 & 3.460 & 20.22 & 1 & 0 & 3 &
1\tabularnewline
Duster 360 & 14.3 & 8 & 360.0 & 245 & 3.21 & 3.570 & 15.84 & 0 & 0 & 3 &
4\tabularnewline
Merc 240D & 24.4 & 4 & 146.7 & 62 & 3.69 & 3.190 & 20.00 & 1 & 0 & 4 &
2\tabularnewline
Merc 230 & 22.8 & 4 & 140.8 & 95 & 3.92 & 3.150 & 22.90 & 1 & 0 & 4 &
2\tabularnewline
Merc 280 & 19.2 & 6 & 167.6 & 123 & 3.92 & 3.440 & 18.30 & 1 & 0 & 4 &
4\tabularnewline
\bottomrule
\end{longtable}

Also a quote using \texttt{\textgreater{}}:

\begin{quote}
``He who gives up {[}code{]} safety for {[}code{]} speed deserves
neither.''
(\href{https://twitter.com/hadleywickham/status/504368538874703872}{via})
\end{quote}


\end{document}
